\documentclass[10pt, a4paper]{article}

\usepackage[T2A]{fontenc}            % внутренняя кодировка  TeX
\usepackage[utf8x]{inputenc}         % кодовая страница документа
\usepackage[english, russian]{babel} % локализация и переносы
\usepackage{indentfirst}   % русский стиль: отступ первого абзаца раздела
\usepackage{textcomp}
\usepackage[margin=0.5in]{geometry}

\title{Правила пользования пассажирским лифтом с автоматическим приводом дверей}
\thispagestyle{empty}

\newcommand{\ffbox}[1]{%
  {%
   \setlength{\fboxsep}{-1\fboxrule}%
   \fbox{\hspace{2.3pt}\strut#1\hspace{2.3pt}}%
  }%
}
\begin{document}

\textbf{
  \mbox{}\hfill Регистрационный номер лифта: RRL1314395\\
  \mbox{}\hfill Ответственный: Чалкин~А.В., тел.~+7~(495)~648~87~77
}

\section*{Правила пользования}

\begin{enumerate}
  \item Для вызова кабины нажмите кнопку, расположенную около лифта. Если вызов принят, на кнопке светится Индикатор.
  \item О прибытии кабины на этаж извещает цифровая или звуковая сигнализация.
  \item После автоматического открытия дверей убедитесь, что кабина лифта находится перед вами, пропустите выходящих пассажиров.
  \item Войдите в кабину, нажмите кнопку нужного Вам этажа, двери закроются автоматически кабина придёт в движение.
  \item При световой и звуковой сигнализации о перегрузках часть пассажиров или груза должны быть удалены из кабины до прекращения
    сигнализации.
  \item Двери автоматические. Если при закрытии дверей кабины Вам необходимо их открыть, а также если двери закрылись, и кабина не пришла
    движение~--- нажмите кнопку с символом \ffbox{\textlangle~||~\textrangle}. Для ускорення закрытия дверей нажмите кнопку
    \ffbox{\textbf{\textrangle~||~\textlangle}}.
  \item При перевозке ребенка в коляске необходимо:
    \begin{itemize}
      \item взять ребенка на руки,
      \item войти в кабину,
      \item ввезти коляску.
    \end{itemize}
    При выходе сначала вывезите коляску, а затем выходите сами с ребенком на руках.
  \item При поездке взрослых с детьми, первыми в кабину лифта должны входить взрослые, а затем дети. При выходе первыми выходят дети.
  \item При поездке с собаками, входя и выходя из кабины, держите их за ошейник.
  \item Допускается перевозить крупногабаритные грузы только в присутствии обслуживающего персонала. При погрузочно-разгрузочных работах
    длительностью более 5 минут необходимо вызвать электромеханика, нажав кнопку «Вызов».
  \item При остановке кабины между этажами нажмите кнопку «Вызов», сообщите о случившимся и выполняйте указання диспетчера.
\end{enumerate}

\section*{Что запрещено}

\begin{itemize}
  \item при остановке кабины между этами пытаться выйти из нее~--- это \textbf{опасно};
  \item пользоваться лифтом детям дошкольного возраста без сопровождения взрослых;
  \item открывать двери лифта вручную или при движении;
  \item проникать шахту или приямок;
  \item курить и бросать мусор в кабине лифта;
  \item перевозить врывоопасные, легко поспламеняющиеся и ядовитые грузы;
  \item пользоваться лифтом, если кабина задымлена и ощущается запах гари;
  \item препятствовать закрытию дверей.
\end{itemize}

\small{
  \noindent
  \mbox{}\hfill грузоподъемность лифта 1000~кг, скорость 1~м/с\\
  \\
  \mbox{}\hfill ООО Лифтэнд Сервис 129301, г. Москва, ул. Касаткина, д. ЗА\\
  \mbox{}\hfill ziplift.ru +7~(495)~134-46-04
}

\vspace*{\fill}
\tiny{
  \mbox{}\hfill PDF можно скачать здесь: https://github.com/johnlepikhin/skandinavia/blob/master/lifts/
}

\end{document}
