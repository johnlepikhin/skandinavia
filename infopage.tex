\documentclass[10pt,landscape,a4paper]{article}
\usepackage[utf8]{inputenc}
\usepackage[english,russian]{babel}
\usepackage[T1]{fontenc}
\usepackage{multicol}
\usepackage{wrapfig}
\usepackage[margin=1cm]{geometry}
\usepackage[framemethod=tikz]{mdframed}
\usepackage{microtype}
\usepackage{pdfpages}
\usepackage{hyperref}
\usepackage{nopageno}
\usepackage{qrcode}

\let\bar\overline

\setlength{\parindent}{0.5cm}

\begin{document}

\makeatletter{
  \renewcommand*\@makefnmark{}
  \footnotetext{Хотите дополнить? Пишите в WhatsApp или Telegram: +7~926~146~23~36}
  \makeatother
}

\section*{Полезная информация\hfill\qrcode{https://github.com/johnlepikhin/skandinavia/releases/download/v0.2.0/infopage.pdf}}

\noindent\makebox[\linewidth]{\rule{0.9\paperwidth}{0.4pt}}

\small
\begin{multicols*}{3}
  \section*{Телефоны}
  \begin{description}
  \item[А101, диспетчерская]\hfill \\ \textbf{+7~495~648-67-77. Круглосуточно}. Потоп, нет электричества, …
  \item[А101, бухгалтерия]\hfill \\ \textbf{+7~495~648-67-77. Вт, чт, 14:00-17:00}. Сумма в платежках, перерасчет, и пр.
  \item[Старший смены по охране района]\hfill \\ \textbf{+7~903~758-27-90}. На улице драка, пьяный спит на скамейке, кража.
  \item[Охрана на первом этаже]\hfill \\ \textbf{+7~903~193-85-03}.
  \item[Участковый]\hfill \\ \textbf{+7~999~010-77-12}. Корольков Станислав Олегович.
  \item[Мосэнергосбыт]\hfill \\ \textbf{+7~499~550-95-50}.
  \end{description}

  \section*{Чаты}

  \begin{description}
  \item[Чат жителей дома в WhatsApp] Познакомьтесь с соседями и вас добавят.
  \item[Чат охраны Скандинавии в Telegram] \href{https://t.me/a101safechat}{@a101safechat}
  \item[Чат жителей Скандинавии в Telegram] \href{https://t.me/gkskandinavia}{@gkskandinavia}
  \end{description}

  \vfill\null
  \columnbreak
  
  \section*{Взаимодействие с УК}
  \subsubsection*{Адрес}
  А101, их бухгалтерию и отдел по работе с клиентами можно найти по адресу ул.~Скандинавский бульвар, д.~6 (в народе «дом-утюг»).
  
  \subsubsection*{Отопление}
  За отопление каждый месяц начисляется одна и ту же сумма. Обещают пересчитать по факту потребления в 2021 году.

  \subsubsection*{Домофон}
  Обслуживанием и подключением домофона занимается интернет-провайдер INT-TEL, +7~967~265-66-79.

  Появилась возможность установить приложение видео-домофона на телефон. Для этого надо поставить приложение
  \href{https://play.google.com/store/apps/details?id=com.basip.app}{BAS-IP Intercom}, а также в УК (см.~адрес выше) получить SIP-номер для
  телефона.

  Видео-домофон как отдельное устройство в квартиру стоит 16000~руб. по состоянию на начало 2020~г. Технически, наш домофон позволяет также
  вход по NFC («приложить телефон»), однако УК никак не может это нам обеспечить.

  \vfill\null
  \columnbreak

  \section*{Частые вопросы}
  \subsubsection*{Как оформить прописку?}

  Для этого надо записаться на прописку на Гос услугах, выбрать МФЦ на ул.~Александры Монаховой, д.~23. До прихода квитка о готовности вас
  принять, можете позвонить +7~495~777-77-77 или сходить туда и узнать текущее расписание МВД т.к. правила у работы у них постоянно
  меняются. Также спросите, как рано стоит прийти чтобы результативно занять очередь.

  Второй вариант~--- МФЦ. Надо предварительно записаться через МФЦ: Каталог услуг $\rightarrow$ запись на приём в центры госуслуг или
  ОИВ/ГБУ МФЦ г.Москвы $\rightarrow$ Регистрационный учет граждан по месту пребывания ТиНАО $\rightarrow$ выбрать удобный центр, указать
  количество людей, котрые будут регистрироваться в графе <<количество дел/объектов>>.

  \subsubsection*{Как сделать ещё одну карточку для домофона?}
  Официальный путь: пойти в УК, где за 500 рублей дадут такую же.

  Неофициальный путь: пойти в дом быта, где делают копии ключей, попросить сделать копию на брелок с карты.

  \subsubsection*{Какое у нас почтовое отделение?}

  108801 ул. Липовый Парк, д.~8, корп.~2

  \subsubsection*{Какие интернет-провайдеры доступны?}

  \href{http://int-tel.net/}{INT-TEL} и \href{https://moscow.rt.ru/}{Ростелеком}.

  \subsubsection*{Ремонт закончили, как отказаться от услуги вывоза мусора?}

  Услуга стоит 1000~руб. в месяц и тихо продлевается бухгалтерией А101 до бесконечности. Чтобы отказаться от неё, необходимо посетить
  бухгалтерию.

\end{multicols*}

\end{document}
